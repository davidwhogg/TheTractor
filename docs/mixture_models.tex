% Copyright 2012 David W. Hogg (NYU).
% All rights reserved.

\documentclass[12pt]{article}
\usepackage{amssymb,amsmath,mathrsfs}

\newcommand{\tmatrix}[1]{\boldsymbol{#1}}
\newcommand{\inverse}[1]{{#1}^{-1}}
\newcommand{\transpose}[1]{{#1}^{\mathsf T}}
\newcommand{\tvector}[1]{\boldsymbol{#1}}
\newcommand{\pos}{\tvector{x}}
\newcommand{\spos}{\tvector{\xi}}
\newcommand{\mean}{\tvector{m}}
\newcommand{\var}{\tmatrix{V}\!}
\newcommand{\Gm}{\tmatrix{G}}
\newcommand{\Hm}{\tmatrix{H}}
\newcommand{\Rm}{\tmatrix{R}}
\newcommand{\uv}{\tvector{u}}
\newcommand{\zero}{\tmatrix{0}}
\newcommand{\identity}{\tmatrix{I}}
\newcommand{\normal}{N}
\newcommand{\given}{\,|\,}
\renewcommand{\star}{\mathrm{star}}
\newcommand{\dev}{\mathrm{dev}}

\begin{document}

\section*{Replacing standard galaxy profiles with \\ mixtures of Gaussians}

\noindent
David W. Hogg \\
\textsl{New York University} \\
\textsl{Max-Planck-Institut f\"ur Astronomie} \\[1ex]
Dustin Lang \\
\textsl{Princeton University Observatory}

\begin{abstract}
Exponential, de~Vaucouleurs, and Sersic profiles are simple and
successful models for fitting two-dimensional images of galaxies.  One
numerical issue encountered in this kind of fitting is the pixel
rendering and convolution (or correlation) of the models with the
telescope point-spread function (PSF); these operations are slow, and
easy to get slightly wrong at small radii.  Here we exploit the
realization that these models can be approximated to arbitrary
accuracy with a mixture (linear sum) of two-dimensional Gaussians.
Mixtures of Gaussians are fast to render, fast to affine-transform,
and fast to convolve with mixture-of-Gaussian PSF models, all at
machine precision.  We present worked examples that can be directly
used in image fitting; we are using them ourselves.  We also advocate
modeling PSFs also as arbitrary mixtures of Gaussians.  Amusingly, in
the optically thin limit, a circularly symmetric mixture-of-Gaussian
two-dimensional model directly implies its own spherically symmetric
three-dimensional de-projection.
\end{abstract}

...Some other authors have worked with mixtures of Gaussians...Very
related work...thinking more generally...Here we want to solve a very
specific set of numerical problems...

...Mixtures of Gaussians sometimes used for PSF fitting...Used in
XD...Other places?...

...Rendering of Sersic profiles (of which exp and dev are types) is
non-trivial at the center...for the same reason convolution is a
bitch...

...If we wanted to approximate these models, what objective function
would we use?  What tolerances are okay?...

...What do we get by running the relevant approximations for exp and
dev?...Can we make a continuous approximation for the Sersic
profiles?...

...How to use these in a real setting...An example from the Tractor...

\begin{eqnarray}\displaystyle
\normal(\pos\given\mean,\var) &\equiv& \frac{1}{2\pi}\,\det(\var)^{-1/2}\,\exp(-\frac{1}{2}\,\transpose{[\pos-\mean]}\cdot\inverse{\var}\cdot[\pos-\mean])
\quad ,
\end{eqnarray}

\begin{eqnarray}\displaystyle
Q_{\exp}(\spos) &=& \sum_{m=1}^{M_g} g_m\,\normal(\spos\given\zero,\Gm_m)
\\
1 &=& \sum_{m=1}^{M_g} g_m
\\
\Gm_m &=& \sigma^2_{\exp,m}\,\identity
\quad ,
\end{eqnarray}

\begin{eqnarray}\displaystyle
Q_{\dev}(\spos) &=& \sum_{m=1}^{M_h} h_m\,\normal(\spos\given\zero,\Hm_m)
\\
1 &=& \sum_{m=1}^{M_h} h_m
\\
\Hm_m &=& \sigma^2_{\dev,m}\,\identity
\quad ,
\end{eqnarray}

\begin{eqnarray}\displaystyle
\psi(\Delta\pos) &=& \sum_{k=1}^K p_k\,\normal(\Delta\pos\given\mean_k,\var_k)
\\
1 &=& \sum_{k=1}^K p_k
\\
I(\pos\given\star,S_s,\pos_s) &=& \sum_{k=1}^K S_s\,p_k\,\normal(\pos\given\pos_s+\mean_k,\var_k)
\quad ,
\end{eqnarray}

\begin{eqnarray}\displaystyle
I(\pos\given\exp,S_g,\pos_g,\var_g) &=& \sum_{k=1}^K \sum_{m=1}^{M_g} S_g\,g_m\,p_k\,\normal(\pos\given\pos_g+\mean_k,\var_{gn}+\var_k)
\\
\var_{gn} &\equiv& \Rm_g\cdot\Gm_m\cdot\transpose{\Rm_g}
\\
\var_g &\equiv& \Rm_g\cdot\transpose{\Rm_g}
\\
\Rm_g &=& \left[a\,\uv_1 , b\,\uv_2 \right]
\quad ,
\end{eqnarray}
where $a$ and $b$ are the major and minor axis lengths of the galaxy
ellipse (in appropriate units) and $\uv_1$ and $\uv_2$ are the
eigenvectors on the sky pointing in the major-axis and minor-axis
directions respectively.  Since implicitly all vectors are
two-dimensional column vectors, this makes $\Rm_g$ a $2\times 2$
affine transformation matrix.

\begin{eqnarray}\displaystyle
I(\pos\given\dev,S_h,\pos_h,\var_h) &=& \sum_{k=1}^K \sum_{m=1}^{M_h} S_h\,h_m\,p_k\,\normal(\pos\given\pos_h+\mean_k,\var_{hn}+\var_k)
\\
\var_{hn} &\equiv& \Rm_h\cdot\Hm_m\cdot\transpose{\Rm_h}
\\
\var_h &\equiv& \Rm_h\cdot\transpose{\Rm_h}
\quad ,
\end{eqnarray}
where...

\end{document}
